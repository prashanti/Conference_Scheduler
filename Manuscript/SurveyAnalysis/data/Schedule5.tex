\documentclass{article}
\usepackage[utf8]{inputenc}
\usepackage{array}
\usepackage{ltxtable, tabularx, longtable,tabu}
\title{Schedule}
\date{}

\begin{document}
\maketitle{}

\section{Saturday, May 23, 8:30 - 9:45}
\begin{longtabu} to \textwidth {lX}
 & \textbf{Session 1} \\ 

8.30 - 8:45 & Identifying Drivers of Island Speciation on an Ancient Tropical Island Using Next Generation Sequencing Data \\ 
 &  \\ 
8:45 - 9:00 & Macroevolutionary consequences of chemical defence in amphibians \\ 
 &  \\ 
9:00 - 9:15 & Functional traits  biomes  and diversification rates in Rhamnaceae \\ 
 &  \\ 
9:15 - 9:30 & Genetic mapping of horizontal stripes in Lake Victoria cichlid fishes  benefits and pitfalls of using of dense linkage mapping using RAD markers \\ 
 &  \\ 
9:30 - 9:45 & A history of arrivals and subsequent diversification in Madagascar  A case study from the myrrh genus  Commiphora Jacq  Burseraceae \\ 
 &  \\ 
 & \textbf{Session 2} \\ 

8.30 - 8:45 & Candidate barrier genes between G firmus and G pennsylvanicus are concentrated on the X chromosome \\ 
 &  \\ 
8:45 - 9:00 & The effects of interspecific hybrid incompatibilities on gene flow during complex speciation \\ 
 &  \\ 
9:00 - 9:15 & Gene flow dynamics between two Indian fruit bats  what can whole genome scans reveal \\ 
 &  \\ 
9:15 - 9:30 & Patterns of gene flow and reproductive isolation in closely related species of mushroom feeding Drosophila \\ 
 &  \\ 
9:30 - 9:45 & Smells like fish species  massively parallel sequencing supports sympatric speciation of coral reef fishes  genus  Haemulon \\ 
 &  \\ 
 & \textbf{Session 3} \\ 

8.30 - 8:45 & Rapid evolution and duplication of a key centromeric protein in Mimulus  a genus with female meiotic drive \\ 
 &  \\ 
8:45 - 9:00 & Search for the evolutionary origin of the brain \\ 
 &  \\ 
9:00 - 9:15 & Centromere associated drive and the maintenance of fitness variation in Mimulus \\ 
 &  \\ 
9:15 - 9:30 & Using fossils  ecology  and molecules to understand the mechanisms shaping dry forest bird species diversity and distributions \\ 
 &  \\ 
9:30 - 9:45 & What ion channel gene duplications can tell us about the origin s  of the nervous system \\ 
 &  \\ 
 & \textbf{Session 4} \\ 

8.30 - 8:45 & Are female mating decisions adaptive when environments vary  A test using natural resource variation \\ 
 &  \\ 
8:45 - 9:00 & Battle of the sexes  may the best fly win in reproduction \\ 
 &  \\ 
9:00 - 9:15 & Male driven evolution of self compatibility in diploid and polyploid Arabidopsis \\ 
 &  \\ 
9:15 - 9:30 & Dissecting an intersexual genetic correlation for fitness using whole genome sequence data \\ 
 &  \\ 
9:30 - 9:45 & Broad Sense  Sexual Conflict  A New Model of Evolution of Resistance to Sexual Violence \\ 
 &  \\ 
 & \textbf{Session 5} \\ 

8.30 - 8:45 & Genetics of Ecological Specialization and Incipient Speciation in an Experimental Population of E coli \\ 
 &  \\ 
8:45 - 9:00 & Constraints on speciation and local adaptation  On the role of variable selection acting among adaptive traits \\ 
 &  \\ 
9:00 - 9:15 & Impacts of anthropogenic disturbance to phenotypic traits under selection in incipient speciation \\ 
 &  \\ 
9:15 - 9:30 & Sexy signals  gamete recognition is a reproductive mechanism and isolating barrier in wild tomato species  Solanum sect Lycopersicon \\ 
 &  \\ 
9:30 - 9:45 & The role of adaptive introgression in a case of incipient speciation in Mimulus \\ 
 &  \\ 
 & \textbf{Session 6} \\ 

8.30 - 8:45 & Rate of resistance evolution in long  and short lived hosts \\ 
 &  \\ 
8:45 - 9:00 & Immune memory drives the evolution of virulence  in an emergent wildlife pathogen \\ 
 &  \\ 
9:00 - 9:15 & Elevational disease distribution in a natural plant pathogen system  Insights from genetic variation in resistance and morphology \\ 
 &  \\ 
9:15 - 9:30 & Life history response to juvenile stress in Apis mellifera \\ 
 &  \\ 
9:30 - 9:45 & Social context of disease resistance  Interactions among social and individual immune defense mechanisms in honey bees \\ 
 &  \\ 
 & \textbf{Session 7} \\ 

8.30 - 8:45 & Eco evo dynamics of mutualism decline in response to nitrogen fertilization \\ 
 &  \\ 
8:45 - 9:00 & Helpful allies or suffocating friends   Evolution in a mutualism constrains adaptive change \\ 
 &  \\ 
9:00 - 9:15 & Resolving relationships in  Cecropieae  Urticaceae   Implications for the evolution of an ant plant mutualism \\ 
 &  \\ 
9:15 - 9:30 & Estimating phenotypic selection in an age structured moose Alces alces population by removing transient fluctuations \\ 
 &  \\ 
9:30 - 9:45 & The Unbearable Lifespan of Beings  A Revision to the Evolutionary Theory of Ageing \\ 
 &  \\ 
 & \textbf{Session 8} \\ 

8.30 - 8:45 & Latitudinal clines in genome wide variation predict host race differences in Rhagoletis pomonella \\ 
 &  \\ 
8:45 - 9:00 & Sexual selection impedes ecological specialization \\ 
 &  \\ 
9:00 - 9:15 & Macro  and microevolutionary perspectives on the evolution of terrestrial egg laying in frogs \\ 
 &  \\ 
9:15 - 9:30 & HOST BIRDS COMBAT CUCKOO MIMICRY BY EVOLVING RECOGNIZABLE EGG PATTERN SIGNATURES \\ 
 &  \\ 
9:30 - 9:45 & Genome wide evidence of genetic associations in eclosion timing in Rhagoletis fruit fly host races \\ 
 &  \\ 
 & \textbf{Session 9} \\ 

8.30 - 8:45 & All roads lead to Rome  in the development of a vestigial eye  Convergent evolution between Eurycea rathbuni and Astyanax mexicanus \\ 
 &  \\ 
8:45 - 9:00 & Differential expression of carotenoid biosynthesis genes may underlie function in gall midges \\ 
 &  \\ 
9:00 - 9:15 & Ecological constraints on sensory systems  Compound eye size in Daphnia is reduced by resource limitation \\ 
 &  \\ 
9:15 - 9:30 & Multicellular evolution in the volvocine algae evolved through genetic permanence of a predator evasion response in unicellular Chlamydomonas \\ 
 &  \\ 
9:30 - 9:45 & The relative importance of primary and secondary resources in adaptive radiation \\ 
 &  \\ 
 & \textbf{Session 10} \\ 

8.30 - 8:45 & Is the general time reversible model bad for phylogenetics \\ 
 &  \\ 
8:45 - 9:00 & The Ruby Seadragon  a spectacular new species of seadragon  Syngnathidae \\ 
 &  \\ 
9:00 - 9:15 & Gene flow and local adaptation at the lower elevation range limit of the montane salamander  Plethodon ouachitae \\ 
 &  \\ 
9:15 - 9:30 & Global Influenza A H3N2 Migration under Changing Prevalence throughout the Year \\ 
 &  \\ 
9:30 - 9:45 & Assessing the intraspecific systematics of the Cotton Mouse  Peromyscus gossypinus  using a highly variable region of the mitochondrial genome \\ 
 &  \\ 
 & \textbf{Session 11} \\ 

8.30 - 8:45 & Transcriptome analyses and differential gene expression in the cone snail Conus miliaris  Effects of  predator prey interactions on venom evolution \\ 
 &  \\ 
8:45 - 9:00 & The extremes of toxin expression variation revealed in two sympatric snake species \\ 
 &  \\ 
9:00 - 9:15 & Expression of venom homologs in the python suggest a model for venom gene recruitment and question the definition of a venom toxin \\ 
 &  \\ 
9:15 - 9:30 & Inferring phylogenetic relationships and understanding venom evolution in cone snails  genus  Conus  using venom duct transcriptomes \\ 
 &  \\ 
9:30 - 9:45 & Spider Transcriptomes Identify Ancient Large Scale Gene Duplication Event and its Role in Silk Gland Evolution \\ 
 &  \\ 
 & \textbf{Session 12} \\ 

8.30 - 8:45 & Genetic diversity of two crustaceans with presumed different reproductive modes in ponds of the Chihuahuan Desert  North America \\ 
 &  \\ 
8:45 - 9:00 & Using genetic variation to infer the comparative demographic history of avian populations in the West Indies \\ 
 &  \\ 
9:00 - 9:15 & Gynodioecy and sex ratio variation within a large network of natural populations \\ 
 &  \\ 
9:15 - 9:30 & Population genetics and phylogeography of wild maize along environmental gradients \\ 
 &  \\ 
9:30 - 9:45 & Optimal sampling of plant populations for ex situ conservation of genetic biodiversity  considering realistic population structure \\ 
 &  \\ 
 & \textbf{Session 13} \\ 

8.30 - 8:45 & Snake genomes provide insight into the molecular evolutionary origins of a phenotypically distinct vertebrate clade \\ 
 &  \\ 
8:45 - 9:00 & The Diversification of LINE Transposable Elements in Vertebrate Genomes  Patterns and Processes \\ 
 &  \\ 
9:00 - 9:15 & WHOLE GENOME RESEQUENCING OF EXPERIMENTAL LINEAGES OF DROSOPHILA MELANOGASTER EXPOSED TO CHRONIC LARVAL MALNUTRITION FOR OVER 150 GENERATIONS \\ 
 &  \\ 
9:15 - 9:30 & Tick tock goes the croc  Three genome drafts indicate slow molecular evolution in crocodilians and provide insight into archosaur evolution \\ 
 &  \\ 
9:30 - 9:45 & Sample sequencing of 40 squamate reptile genomes reveals extensive evolutionary dynamics of genomic repeat element landscapes \\ 
 &  \\ 
 & \textbf{Session 14} \\ 

8.30 - 8:45 & Accelerated Anchored Phylogenomics  A new paradigm enables 10 fold increase in throughput and 10 fold decrease in cost for phylogenomics \\ 
 &  \\ 
8:45 - 9:00 & Anchored Phylogenomics in Angiosperms  Maximizing Data Compatibility Through Coordinated Locus Selection \\ 
 &  \\ 
9:00 - 9:15 & Anchored Phylogenomics in Angiosperms  Utility across Angiosperms at Shallow Levels \\ 
 &  \\ 
9:15 - 9:30 & Phylogenomics and the Evolution of Paedomorphism in the Cyprinidae \\ 
 &  \\ 
9:30 - 9:45 & The root of the flowering plants  re re revisited \\ 
 &  \\ 
\end{longtabu}
\section{Saturday, May 23, 10:30 - 11:45}
\begin{longtabu} to \textwidth {lX}
 & \textbf{Session 1} \\ 

10:00 - 10:15 & The impact of breeding protocol on inbreeding, genetic diversity, and adaptation to captivity as measured in experimental populations of deer mice \\ 
 &  \\ 
10:15 - 10.30 & Genetic population structure of Blanding’s turtle (Emydoidea blandingii) in southeastern Wisconsin \\ 
 &  \\ 
10.30 - 10.45 & Fine-scale sampling reveals discordant patterns of genetic variation in Mojave lizard populations \\ 
 &  \\ 
10:45 - 11:00 & ddRAD-seq analyses of population structure in brood parasitic indigobirds (Vidua spp.) \\ 
 &  \\ 
 & \textbf{Session 2} \\ 

10:00 - 10:15 & Male-beneficial genotypes harbor deleterious genetic architecture \\ 
 &  \\ 
10:15 - 10.30 & Understanding epistasis and gene networks in complex traits: An analysis of aggression in a model system \\ 
 &  \\ 
10.30 - 10.45 & Opposing genotype-by-environment interactions and the maintenance of a genetic color polymorphism in a livebearing fish \\ 
 &  \\ 
10:45 - 11:00 & Payoffs and Tradeoffs \\ 
 &  \\ 
11:00 - 11:15 & Comparative quantitative genetics of the pelvis in four species of rodents: Evolution of the genetic and phenotypic covariance structure \\ 
 &  \\ 
 & \textbf{Session 3} \\ 

10:00 - 10:15 & Comparative methods for evaluating the evolutionary history of function-valued traits: a case study of salt tolerance in wild Helianthus (sunflowers) \\ 
 &  \\ 
10:15 - 10.30 & Metrics for comparing the fit of time trees to the fossil record \\ 
 &  \\ 
10.30 - 10.45 & Use of principal component analysis in species delimitation leads to (precise?) underestimation of species numbers \\ 
 &  \\ 
10:45 - 11:00 & Non-null effects of a null range: Exploring parameter estimation in the dispersal-extinction-cladogenesis model \\ 
 &  \\ 
11:00 - 11:15 & Molecular species delimitation methods recover most song delimited cicada species in the European Cicadetta montana complex \\ 
 &  \\ 
 & \textbf{Session 4} \\ 

10:00 - 10:15 & Bayesian estimation of phylogenetic information content and its implications for site-stripping \\ 
 &  \\ 
10:15 - 10.30 & Phylogenetic comparative biology and morphometrics collide: PIC, PGLS, and the challenge of high-dimensional data \\ 
 &  \\ 
10.30 - 10.45 & Using networks of topologies and bipartitions to explore, quantify, and summarize phylogenetic tree space \\ 
 &  \\ 
10:45 - 11:00 & Open Tree of Life version 1.0: a comprehensive and easily-updated tree of life \\ 
 &  \\ 
11:00 - 11:15 & Estimating Evolutionary Parameters and Full Length Haplotypes Simultaneously Using Short-Read Sequences Derived from Genetically Variable Populations \\ 
 &  \\ 
 & \textbf{Session 5} \\ 

10:00 - 10:15 & Measuring pollinator-mediated selection with various fitness components: A review and a lesson from Linum pubescens \\ 
 &  \\ 
10:15 - 10.30 & “Fitness” has at least three incommensurable dimensions:  growth, efficiency, and competitiveness \\ 
 &  \\ 
10.30 - 10.45 & The environmental determinants of natural selection \\ 
 &  \\ 
10:45 - 11:00 & Fitness functions and distributions: the shape of things to come \\ 
 &  \\ 
 & \textbf{Session 6} \\ 

11:00 - 11:15 & Rapid diversification and secondary sympatry in an island bird lineage (Aves: Todiramphus) \\ 
 &  \\ 
10:00 - 10:15 & Early burst in ecological radiation of birds \\ 
 &  \\ 
10:15 - 10.30 & Assembly of the New World oscine passerine fauna \\ 
 &  \\ 
10.30 - 10.45 & Avian evolutionary history in the southern Neotropics: complex and varied patterns of diversification \\ 
 &  \\ 
10:45 - 11:00 & A reappraisal of the productivity hypothesis for North American bird assemblages \\ 
 &  \\ 
 & \textbf{Session 7} \\ 

10:00 - 10:15 & An inside-out origin of the eukaryotic cell \\ 
 &  \\ 
10:15 - 10.30 & Predator-induced facultative group formation in Chlamydomonas depends on life history traits and the groups can be chimaeric. \\ 
 &  \\ 
10.30 - 10.45 & The evolution of multicellularity as a key innovation for adaptive radiation in experimental microcosms \\ 
 &  \\ 
10:45 - 11:00 & Scaffolding the origin of multicellular evolvability \\ 
 &  \\ 
11:00 - 11:15 & The evolution of life cycle gene expression in the Volvocine algae: toward a molecular understanding of multicellular evolution \\ 
 &  \\ 
 & \textbf{Session 8} \\ 

10:00 - 10:15 & Whole Genome Sequence of the Behaviorally Polymorphic White-Throated Sparrow 2: Population Genomics \\ 
 &  \\ 
10:15 - 10.30 & Whole Genome Sequence of the Behaviorally Polymorphic White-Throated Sparrow 1: Mapping Genes for Socio-genomics \\ 
 &  \\ 
10.30 - 10.45 & Method to identify small-scale gene transpositions in rearranged genomes. \\ 
 &  \\ 
10:45 - 11:00 & Ancient duplication of vomeronasal receptor class 1 (V1R) genes in lemurs \\ 
 &  \\ 
11:00 - 11:15 & Caught in the Crossfire: Genes tangled up in host-mediated transposable element defense \\ 
 &  \\ 
 & \textbf{Session 9} \\ 

10:00 - 10:15 & Physiological adaptation of thermal sensitivity of Colias larvae in response to climate change \\ 
 &  \\ 
10:15 - 10.30 & Climate variability may limit evolutionary adaptation to climate change in montane and alpine butterflies \\ 
 &  \\ 
10.30 - 10.45 & Genomics of adaptation to altitude in Mus. \\ 
 &  \\ 
10:45 - 11:00 & Modeling and detecting biological responses to climate changeweather variability using first-principles of physiology and estimates of food resource" \\ 
 &  \\ 
11:00 - 11:15 & Target Enrichment of Ultraconserved Elements in Sky Island Frogs of the Brazilian Atlantic Rainforest \\ 
 &  \\ 
 & \textbf{Session 10} \\ 

10:00 - 10:15 & Clonal genotype influences behavior and response to predators in invasive New Zealand mud snails (Potamopyrgus antipodarum) \\ 
 &  \\ 
10:15 - 10.30 & The Origin of South Asian Red Rice and Weed Competitiveness \\ 
 &  \\ 
10.30 - 10.45 & The Trojan Female Technique – A novel approach for pest population control \\ 
 &  \\ 
10:45 - 11:00 & Baker’s General Purpose Genotype: Are highly tolerant weeds also the most fit? \\ 
 &  \\ 
11:00 - 11:15 & More than one way to evolve a weed:  Southeast Asian weedy rice population genetics and US weedy rice QTL mapping studies based on Genotyping-by-Seque \\ 
 &  \\ 
 & \textbf{Session 11} \\ 

10:00 - 10:15 & Evolution of leaf defenses in relation to environment and leaf economics across the genus Helianthus \\ 
 &  \\ 
10:15 - 10.30 & A within-species comparative method reveals reproductive character displacement to a geographic mosaic of interspecific interactions \\ 
 &  \\ 
10.30 - 10.45 & Shade avoidance and Brassica rapa leaf development: Bayesian modeling and QTL analysis allows for predicting phenotypes from genotypes \\ 
 &  \\ 
10:45 - 11:00 & The evolutionary origin and development of adipose fins, exploring novelty in vertebrate appendages \\ 
 &  \\ 
11:00 - 11:15 & Fins as highly integrated building blocks promoting fish morphological disparity \\ 
 &  \\ 
 & \textbf{Session 12} \\ 

10:00 - 10:15 & The evolutionary importance of male and mutual mate choice \\ 
 &  \\ 
10:15 - 10.30 & Shifting lines in the sand: determinants of spatiotemporally dynamic opportunities for sexual selection in a polygynous mammal \\ 
 &  \\ 
10.30 - 10.45 & The Role of Selection in the Rapid Evolution of Reproductive Genes: Are we using the correct null hypothesis? \\ 
 &  \\ 
10:45 - 11:00 & Sexual selection and sperm competition in a widespread dung fly, Sepsis punctum (Diptera: Sepsidae) \\ 
 &  \\ 
11:00 - 11:15 & Partitioning additive and nonadditive genetic effects on offspring quality in a broadcast spawning marine invertebrate \\ 
 &  \\ 
 & \textbf{Session 13} \\ 

10:00 - 10:15 & Spider Transcriptomes Identify Ancient Large-Scale Gene Duplication Event and its Role in Silk Gland Evolution \\ 
 &  \\ 
10:15 - 10.30 & The genomics of adaptation and divergence in a wild sunflower \\ 
 &  \\ 
10.30 - 10.45 & Natural selection maintains high diversity in candidate genes underlying local adaptation to climate: evidence from whole-transcriptome sequencing \\ 
 &  \\ 
10:45 - 11:00 & Comparative transcriptomics identifies the gene repertoires underlying functional differentiation of spider silk glands \\ 
 &  \\ 
 & \textbf{Session 14} \\ 

10:00 - 10:15 & Where are all old fungal ectomycorrhizal lineages? \\ 
 &  \\ 
10:15 - 10.30 & Facultative endohyphal bacterial symbionts alter phenotypes of fungal endophyte hosts \\ 
 &  \\ 
10.30 - 10.45 & Asymmetric host resources affect mycorrrhizal responses to host relatedness \\ 
 &  \\ 
10:45 - 11:00 & An Investigation of Bacterial and Fungal Symbionts of the Planthoppers (Hemiptera: Fulgoroidea) \\ 
 &  \\ 
11:00 - 11:15 & Disentangling the coevolutionary histories of animal gut microbiota \\ 
 &  \\ 
\end{longtabu}

\end{document}