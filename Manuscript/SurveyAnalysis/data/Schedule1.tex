\documentclass{article}
\usepackage[utf8]{inputenc}
\usepackage{array}
\usepackage{ltxtable, tabularx, longtable,tabu}
\title{Schedule 1}
\date{}

\begin{document}
\maketitle{}
\section{Saturday, May 23, 8:30 - 9:45}
\begin{longtabu} to \textwidth {lX}
 & \textbf{Session 1} \\ 

8.30 - 8:45 & Evolution of a mating preference for a trait used in intrasexual competition in genetically monogamous populations \\ 
 &  \\ 
8:45 - 9:00 & Sexually selected traits and genotype by environment interactions \\ 
 &  \\ 
9:00 - 9:15 & The effect of sampling bias on the heritability of preference and the strength of  sexual selection \\ 
 &  \\ 
9:15 - 9:30 & The contribution of genes and the environment on traits important for pre and post copulatory reproductive success in the cactus bug  Narnia femorata \\ 
 &  \\ 
9:30 - 9:45 & A novel application of proteomics to quantify adaptive responses to sperm competition \\ 
 &  \\ 
 & \textbf{Session 2} \\ 

8.30 - 8:45 & Evolutionary genetics of the selfish Segregation Distorter complex \\ 
 &  \\ 
8:45 - 9:00 & Phylogeographic model selection using approximated likelihoods \\ 
 &  \\ 
9:00 - 9:15 & Experimental Evolution of Increased Size and Complexity of Anabaena variabilis \\ 
 &  \\ 
9:15 - 9:30 & Early Evolution of the Genetic Basis for Soma in the Volvocine Green Algae \\ 
 &  \\ 
9:30 - 9:45 & Using linked microsatellites to infer basic population decline parameters  a case study on a Mexican relict spruce \\ 
 &  \\ 
 & \textbf{Session 3} \\ 

8.30 - 8:45 & Life history effects and demographic consequences of interacting QTL for flowering and seed dormancy in Arabidopsis thaliana \\ 
 &  \\ 
8:45 - 9:00 & The evolution of host perception in parasitic plants of the Orobanchaceae \\ 
 &  \\ 
9:00 - 9:15 & Mechanisms for the evolution of seasonal timing in incipient species of Ostrinia moths \\ 
 &  \\ 
9:15 - 9:30 & Natural variation in seed germination speed of Arabidopsis thaliana  complex genetic architecture and response to strong selection \\ 
 &  \\ 
9:30 - 9:45 & Plasticity of seed dormancy compensates for differences in dispersal timing \\ 
 &  \\ 
 & \textbf{Session 4} \\ 

8.30 - 8:45 & Evolutionary history and traits  not invasive status  influences community assembly \\ 
 &  \\ 
8:45 - 9:00 & Anchored phylogenomics and transcriptomics  comparisons between two next gen data sets used for estimating deep level relationships in Lepidoptera \\ 
 &  \\ 
9:00 - 9:15 & Transcriptome phylogeny and evolution of host chemical sequestration within the lichen moths  Insecta  Lepidoptera  Erebidae \\ 
 &  \\ 
9:15 - 9:30 & Sea slugs have their cake and eat it too  a phylogenetic analysis of sponge eating nudibranchs and the defense chemicals they take and reuse \\ 
 &  \\ 
9:30 - 9:45 & Darwin s conundrum revisited  does phylogenetic distance predict invasibility \\ 
 &  \\ 
 & \textbf{Session 5} \\ 

8.30 - 8:45 & Population determinants of persistence  migration  and colonization success in a plant metapopulation \\ 
 &  \\ 
8:45 - 9:00 & Genetic and phenotypic divergence in an island bird  isolation by distance  by colonization or by adaptation \\ 
 &  \\ 
9:00 - 9:15 & Speciation and chemical differentiation in the aposematic and mimetic butterflies  Melinaea \\ 
 &  \\ 
9:15 - 9:30 & Computationally efficient estimation of the number of founders for colonized populations \\ 
 &  \\ 
9:30 - 9:45 & Genetic and environmental contributions to a divergent plumage trait in barn swallows \\ 
 &  \\ 
 & \textbf{Session 6} \\ 

8.30 - 8:45 & Cardiac myopathy and flight performance in starvation selected Drosophila  or the case of the All American flies \\ 
 &  \\ 
8:45 - 9:00 & Genome wide scans for signals of molecular adaptation in polar bear \\ 
 &  \\ 
9:00 - 9:15 & Evolutionary genetics of pigmentation variation in natural populations of Drosophila melanogaster \\ 
 &  \\ 
9:15 - 9:30 & The evolution and transcriptional connectivity of genes underlying ant division of labor \\ 
 &  \\ 
9:30 - 9:45 & The genetic architecture of natural variation in abdominal pigmentation of Drosophila melanogaster females \\ 
 &  \\ 
 & \textbf{Session 7} \\ 

8.30 - 8:45 & How Nonadditivity of Fitness Impacts Alters Selection for Resistance in a Multiple Herbivore Community \\ 
 &  \\ 
8:45 - 9:00 & Episodic nucleotide substitutions in seasonal influenza virus H3N2 can be explained by stochastic genealogical process without positive selection \\ 
 &  \\ 
9:00 - 9:15 & Evolution of elemental composition in E coli under carbon and nitrogen limitation \\ 
 &  \\ 
9:15 - 9:30 & Metagenomic analysis of a ssDNA viral community \\ 
 &  \\ 
9:30 - 9:45 & Endogenous hepdnaviruses  bornaviruses and circoviruses in snakes \\ 
 &  \\ 
 & \textbf{Session 8} \\ 

8.30 - 8:45 & A single gene affects both ecological divergence and mate choice in Drosophila \\ 
 &  \\ 
8:45 - 9:00 & Identifying genes affecting both adaptive divergence and reproductive isolation in Howea palms from Lord Howe Island using RNA Seq \\ 
 &  \\ 
9:00 - 9:15 & Gene family evolution and functional plasticity following whole genome duplication events in plants \\ 
 &  \\ 
9:15 - 9:30 & Genomic imprints of freshwater transitions in the alewife  Alosa pseudoharengus \\ 
 &  \\ 
9:30 - 9:45 & Phylogenomics reveals rapid and complex evolutionary divergence and speciation in wild Solanum \\ 
 &  \\ 
 & \textbf{Session 9} \\ 

8.30 - 8:45 & A genomic selection component experiment in Mimulus guttatus \\ 
 &  \\ 
8:45 - 9:00 & Quantification of coiling patterns in gastropod shells and evaluation of functional traits \\ 
 &  \\ 
9:00 - 9:15 & Looking for evolutionary history of nematodes on the beach to better manage their populations in the fields \\ 
 &  \\ 
9:15 - 9:30 & When field experiments yield unexpected results  lessons learned from measuring selection in White Sands lizards \\ 
 &  \\ 
9:30 - 9:45 & Survival of the fattest  Indices of body condition do not predict fitness in the brown anole  Anolis sagrei \\ 
 &  \\ 
 & \textbf{Session 10} \\ 

8.30 - 8:45 & Phylogenomics shows multiple human infectious lineages of Trypanosoma brucei \\ 
 &  \\ 
8:45 - 9:00 & Microbiome Diversity and Dynamics under Neutral and Selective Models \\ 
 &  \\ 
9:00 - 9:15 & Description of a Novel Genetic Marker for Species Identification of Freshwater Mussel Larvae Recovered from Naturally Infested Fish Hosts \\ 
 &  \\ 
9:15 - 9:30 & The diversity and evolution of the primate skin microbiome  how different are humans from our closest relatives \\ 
 &  \\ 
9:30 - 9:45 & Distribution  specificity and horizontal transmission of microbial symbionts in army ant colonies \\ 
 &  \\ 
 & \textbf{Session 11} \\ 

8.30 - 8:45 & Genes vs culture  song variation across an avian hybrid zone \\ 
 &  \\ 
8:45 - 9:00 & A simple two locus hybrid incompatibility underlies inviability between sympatric Mimulus species \\ 
 &  \\ 
9:00 - 9:15 & A phylogenetic model for measuring departures from the mutation selection balance \\ 
 &  \\ 
9:15 - 9:30 & The origin of species by means of Dobzhansky Muller incompatibilities \\ 
 &  \\ 
9:30 - 9:45 & Two locus hybrid incompatibilities and the introgression of adaptive alleles \\ 
 &  \\ 
 & \textbf{Session 12} \\ 

8.30 - 8:45 & FlatNJ  A novel network based approach to visualize evolutionary and biogeographical relationships \\ 
 &  \\ 
8:45 - 9:00 & Information flow through dominance network in social insect colonies \\ 
 &  \\ 
9:00 - 9:15 & PASTA  A new method to co estimate alignments and trees  even on ultra large datasets  with high accuracy and speed \\ 
 &  \\ 
9:15 - 9:30 & RNAseq analysis elucidate early responses to infection in scleractinian corals \\ 
 &  \\ 
9:30 - 9:45 & Reduced specialization and modularity in an intimate mutualism diversifying on young oceanic islands \\ 
 &  \\ 
 & \textbf{Session 13} \\ 

8.30 - 8:45 & The evolution of fur colour  a marsupial perspective \\ 
 &  \\ 
8:45 - 9:00 & The genetic architecture of local adaptation at fine spatial scales   a case study of three montane conifer species \\ 
 &  \\ 
9:00 - 9:15 & Using pooled sequencing and whole genome environmental association analyses to study local adaptation in three Alpine Brassicaceae species \\ 
 &  \\ 
9:15 - 9:30 & Incongruence among classes of markers and data types in supermatrices  implications for phylogenomics and Drosophila evolution \\ 
 &  \\ 
9:30 - 9:45 & Local adaptation to climate within a tree species range  the case of sugar pine  Pinus lambertiana \\ 
 &  \\ 
 & \textbf{Session 14} \\ 

8.30 - 8:45 & Lizard scales in an adaptive radiation  variation of scale number follows climatic and structural habitat diversity in Anolis lizards \\ 
 &  \\ 
8:45 - 9:00 & Who Are the Fathers  Characterizing Hybrid Origins of Parthenogenetic Aspidoscelis Lizards \\ 
 &  \\ 
9:00 - 9:15 & Diversification and Speciation in the Ethiopian Highlands  Insights from a Radiation of Endemic Frogs \\ 
 &  \\ 
9:15 - 9:30 & Dynamic gradients of river systems mediating dispersal and vicariance of fishes \\ 
 &  \\ 
9:30 - 9:45 & Convergent evolution of alternative developmental trajectories associated with diapause in African and South American killifish \\ 
 &  \\ 
\end{longtabu}
\section{Saturday, May 23, 10:30 - 11:45}
\begin{longtabu} to \textwidth {lX}
 & \textbf{Session 1} \\ 

10:00 - 10:15 & Skull integration and modularity in five toad species of the Rhinella granulosa group \\ 
 &  \\ 
10:15 - 10.30 & Exceptional avian herbivores: Multiple origins of herbivory in the bird order Anseriformes and its correlation with beak shape and body mass \\ 
 &  \\ 
10.30 - 10.45 & Lizard scales in an adaptive radiation: non-random variation of scale size follows climatic and structural habitat diversity in Anolis lizards \\ 
 &  \\ 
10:45 - 11:00 & Quantification of Coiling Patterns in Gastropod Shells and Evaluation of Functional Traits \\ 
 &  \\ 
11:00 - 11:15 & Taking many-to-one to the next level: decoupled evolution in an ultrafast prey capture mechanism \\ 
 &  \\ 
 & \textbf{Session 2} \\ 

10:00 - 10:15 & Developmental Mechanisms for Novel Morphological Evolution: Origin and Diversification of the Avian Skull \\ 
 &  \\ 
10:15 - 10.30 & Late evolutionary origin of modern bird flight inferred from shoulder allometry \\ 
 &  \\ 
10.30 - 10.45 & Neotenous feather replacement facilitates loss of flight in birds \\ 
 &  \\ 
10:45 - 11:00 & A new island rule for birds: evolution towards flightlessness \\ 
 &  \\ 
11:00 - 11:15 & PHYLOGENY AND FORELIMB DISPARITY IN WATERBIRDS \\ 
 &  \\ 
 & \textbf{Session 3} \\ 

10:00 - 10:15 & C. elegans harbors pervasive cryptic genetic variation for embryogenesis \\ 
 &  \\ 
10:15 - 10.30 & Reef-specific patterns of osmotic response in larval and adult eastern oysters, Crassostrea virginica, from a single estuary \\ 
 &  \\ 
10.30 - 10.45 & Breaking the mold: the effects of mutations on phenotypic covariation in the fruit fly wing \\ 
 &  \\ 
10:45 - 11:00 & Genetic architecture of rapid and extreme body size evolution in an island population of house mice \\ 
 &  \\ 
11:00 - 11:15 & Environmental effects on genetic covariances \\ 
 &  \\ 
 & \textbf{Session 4} \\ 

10:00 - 10:15 & Conserved Core Genes are under Positive Selection in a Long-Term Escherichia coli Evolution Experiment \\ 
 &  \\ 
10:15 - 10.30 & Intragenic epistasis on adaptive dynamics at the gene couch potato \\ 
 &  \\ 
10.30 - 10.45 & Comparative genomics sheds light on the evolution and function of the Highly Iterative Palindrome -1 motif in Cyanobacteria \\ 
 &  \\ 
10:45 - 11:00 & Functional analysis of the B gene homolog PISTILLATA reveals novel regulatory interactions controlling stamen identity in Aquilegia coerulea \\ 
 &  \\ 
11:00 - 11:15 & Invade, co-opt, and swap:  Evolution of G1/S cell cycle control in Fungi and other eukaryotes \\ 
 &  \\ 
 & \textbf{Session 5} \\ 

10:00 - 10:15 & Increased egg viability, male mating ability and mating frequency evolve in populations of D. melanogaster selected for resistance to cold shok \\ 
 &  \\ 
10:15 - 10.30 & The evolution of fur colour: a marsupial perspective \\ 
 &  \\ 
10.30 - 10.45 & Survival in a cutthroat world: experimental estimation of natural selection on stickleback armor. \\ 
 &  \\ 
10:45 - 11:00 & Is it time to abandon the holey fitness landscape metaphor? \\ 
 &  \\ 
11:00 - 11:15 & When field experiments yield unexpected results: Lessons learned from measuring selection in White Sands lizards \\ 
 &  \\ 
 & \textbf{Session 6} \\ 

10:00 - 10:15 & Can intralocus sexual conflict explain the maintenance of alternative reproductive tactics? \\ 
 &  \\ 
10:15 - 10.30 & The multifaceted role of mating system on genome evolution. \\ 
 &  \\ 
10.30 - 10.45 & Differential gene expression in ovarian tissue of sexual vs. asexual freshwater snails \\ 
 &  \\ 
10:45 - 11:00 & Congruent phenotypic and transcriptomic responses to testosterone in both sexes:implications for the evolution of endocrine-mediated sexual dimorphism \\ 
 &  \\ 
11:00 - 11:15 & Genetics of polymorphic male-male copulatory behavior in C. elegans \\ 
 &  \\ 
 & \textbf{Session 7} \\ 

10:00 - 10:15 & How to train your symbionts: the dynamics of domestication \\ 
 &  \\ 
10:15 - 10.30 & Experimental evolution of reduced antagonism: a role for host-parasite \\ 
 &  \\ 
10.30 - 10.45 & Examining the presence of a geographic mosaic of coevolution in the walnut aphid biological control system \\ 
 &  \\ 
10:45 - 11:00 & Evolving virulence and defense in a symbiotic community. \\ 
 &  \\ 
11:00 - 11:15 & How Nonadditivity of Fitness Impacts Alters Selection for Resistance in a Multiple-Herbivore Community \\ 
 &  \\ 
 & \textbf{Session 8} \\ 

10:00 - 10:15 & Genomic response to 30-years of selection for increased lifespan reveals increased immunity as correlated trait \\ 
 &  \\ 
10:15 - 10.30 & EVOLUTION OF INCREASED ADULT LONGEVITY IN DROSOPHILA MELANOGASTER POPULATIONS AS CORRELATED RESPONSE FOR ADAPTATION TO LARVAL CROWDING \\ 
 &  \\ 
10.30 - 10.45 & Life-history, Selection and Effective Population Size shaping Evolution during Colonization Lessons from Drosophila melanogaster. \\ 
 &  \\ 
10:45 - 11:00 & Genetic basis of ageing evolution under differential extrinsic mortality in a nematode \\ 
 &  \\ 
 & \textbf{Session 9} \\ 

11:00 - 11:15 & Interactions between host phylogeny and biogeography structure sponge-associated microbial communities \\ 
 &  \\ 
10:00 - 10:15 & Host evolution and ecology govern community assembly of the gut microbiome in lemurs \\ 
 &  \\ 
10:15 - 10.30 & Holarctic biogeography of a widespread host-symbiont association \\ 
 &  \\ 
10.30 - 10.45 & Exploring patterns of symbiont diversity in natural pea aphid populations \\ 
 &  \\ 
10:45 - 11:00 & Environmental context matters: the impact of microbial symbiont on invasive insect host Megacopta cribraria is mediated by host plant \\ 
 &  \\ 
 & \textbf{Session 10} \\ 

10:00 - 10:15 & Auto-toxicity the evolution of the muscular voltage-gated sodium channel in Phyllobates poison frogs \\ 
 &  \\ 
10:15 - 10.30 & A native root herbivore drives the evolution of defensive latex metabolites in nature \\ 
 &  \\ 
10.30 - 10.45 & Transcriptome phylogeny and evolution of host chemical sequestration within the lichen moths (Insecta: Lepidoptera: Erebidae) \\ 
 &  \\ 
10:45 - 11:00 & Genetic basis of alkaloid resistance in harlequin toads and poison frogs \\ 
 &  \\ 
11:00 - 11:15 & Sea slugs have their cake and eat it too: a phylogenetic analysis of sponge-eating nudibranchs and the defense chemicals they take and reuse \\ 
 &  \\ 
 & \textbf{Session 11} \\ 

10:00 - 10:15 & Is self-pollination an evolutionary dead end? The evolution of mating systems in Erythranthe section Paradantha (Phrymaceae) \\ 
 &  \\ 
10:15 - 10.30 & How does pollination mutualism affect the evolution of prior self-fertilization? A model \\ 
 &  \\ 
10.30 - 10.45 & The breakdown of self-incompatibility in a range expansion \\ 
 &  \\ 
10:45 - 11:00 & Does separation between sexual organs affect mating?  A case study from the alpine primrose Primula halleri \\ 
 &  \\ 
 & \textbf{Session 12} \\ 

10:00 - 10:15 & Why is Madagascar special? Diversification patterns in pelican spiders (Archaeidae) \\ 
 &  \\ 
10:15 - 10.30 & Diversification and Speciation in the Ethiopian Highlands: Insights from a Radiation of Endemic Frogs \\ 
 &  \\ 
10.30 - 10.45 & The total inventory of Cuatro Ciénegas (Coahuila, Mexico): Patterns and evolutionary causes of high diversity of an oligotrophic aquatic ecosystem \\ 
 &  \\ 
10:45 - 11:00 & The tangled evolutionary histories of Madagascar's small mammals \\ 
 &  \\ 
11:00 - 11:15 & Zoogeography of genus Salvelinus in Kamchatka Peninsula \\ 
 &  \\ 
 & \textbf{Session 13} \\ 

10:00 - 10:15 & Recent divergence in fungal populations \\ 
 &  \\ 
10:15 - 10.30 & Multi-trait divergence driven by predation environment causes immigrant inviability in Brachyrhaphis fishes \\ 
 &  \\ 
10.30 - 10.45 & Mechanisms for the evolution of seasonal timing in incipient species of Ostrinia moths \\ 
 &  \\ 
10:45 - 11:00 & Genomic divergence of putatively adaptive genes along an altitudinal gradient in the common yellow monkeyflower, Mimulus guttatus. \\ 
 &  \\ 
11:00 - 11:15 & An intraspecific gradient from C3 to C4 photosynthesis \\ 
 &  \\ 
 & \textbf{Session 14} \\ 

10:00 - 10:15 & Invasion and hybridization of the highly aggressive introduced reed, Phragmites australis, in the York River watershed \\ 
 &  \\ 
10:15 - 10.30 & Life History and Behavior in a Primate Hybrid Zone \\ 
 &  \\ 
10.30 - 10.45 & What is the link between transmission ratio distortion and sterility in Mimulus hybrids? \\ 
 &  \\ 
10:45 - 11:00 & Paternal learning of a phenotype-matching trait promotes speciation at secondary contact, but not the spread of a new local adaptation \\ 
 &  \\ 
\end{longtabu}


\end{document}
