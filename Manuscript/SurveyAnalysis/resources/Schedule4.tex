\documentclass{article}
\usepackage[utf8]{inputenc}
\usepackage{array}
\usepackage{ltxtable, tabularx, longtable,tabu}
\title{Schedule}
\date{}

\begin{document}
\maketitle{}

\section{Saturday, May 23, 8:30 - 9:45}
\begin{longtabu} to \textwidth {lX}
 & \textbf{Session 1} \\ 

8.30 - 8:45 & Male-limited evolution shapes sexual dimorphism in longevity \\ 
 &  \\ 
8:45 - 9:00 & WHOLE GENOME RESEQUENCING OF EXPERIMENTAL LINEAGES OF DROSOPHILA MELANOGASTER EXPOSED TO CHRONIC LARVAL MALNUTRITION FOR OVER 150 GENERATIONS \\ 
 &  \\ 
9:00 - 9:15 & Experimental removal of parental care leads to the evolution of reduced offspring dependence in the burying beetle, Nicrophorus vespilloides \\ 
 &  \\ 
9:15 - 9:30 & Conflict increases cooperation between microbial species \\ 
 &  \\ 
9:30 - 9:45 & Evolution of ecological dominance of yeast species in high-sugar environments \\ 
 &  \\ 
 & \textbf{Session 2} \\ 

8.30 - 8:45 & Gymnosperm plastomes reveal rampant rearrangements and the retention of ndh pseudogenes in the Pinaceae \\ 
 &  \\ 
8:45 - 9:00 & Tracing the dispersal of the baobab Adansonia digitata (Malvaceae: Bombacoideae) from Africa to the Indian Ocean region: An interdisciplinary approach \\ 
 &  \\ 
9:00 - 9:15 & Exon capture phylogenomics of Australian skinks \\ 
 &  \\ 
9:15 - 9:30 & Pyrosequencing of surface waters in the English Channel reveals novel early-diverging fungal diversity \\ 
 &  \\ 
9:30 - 9:45 & Using genome-wide RAD markers to resolve character evolution and species history in Nymphalid butterflies \\ 
 &  \\ 
 & \textbf{Session 3} \\ 

8.30 - 8:45 & It all adds up: The genetics of thermal reaction norm variation for antibiotic resistance \\ 
 &  \\ 
8:45 - 9:00 & Rate of resistance evolution in long- and short-lived hosts \\ 
 &  \\ 
9:00 - 9:15 & Genomic Variations Associated with Gonococcal Antimicrobial Resistance \\ 
 &  \\ 
9:15 - 9:30 & Elevational disease distribution in a natural plant pathogen system: Insights from genetic variation in resistance and morphology. \\ 
 &  \\ 
 & \textbf{Session 4} \\ 

8.30 - 8:45 & Evidence for rapid adaptation to an environmental contaminant in a model songbird \\ 
 &  \\ 
8:45 - 9:00 & Functional analysis of adaptive evolution of ADH in Drosophila \\ 
 &  \\ 
9:00 - 9:15 & Context-dependent effects of sampling design and demographic history on genome scans for local adaptation \\ 
 &  \\ 
9:15 - 9:30 & Climate change and the high fitness costs of seasonal camouflage mismatch in snowshoe hares \\ 
 &  \\ 
9:30 - 9:45 & Constraints on speciation and local adaptation: On the role of variable selection acting among adaptive traits. \\ 
 &  \\ 
 & \textbf{Session 5} \\ 

8.30 - 8:45 & Plant defenses in the genus Physalis are not constrained by trade-offs between constitutive and induced defenses \\ 
 &  \\ 
8:45 - 9:00 & The impact of inbreeding depression on the evolution of herbicide resistance in the agricultural crop weed, Ipomoea purpurea \\ 
 &  \\ 
9:00 - 9:15 & Direct and correlated responses to artificial selection for herbicide resistance in Ipomoea purpurea: Divergence in traits and the transcriptome \\ 
 &  \\ 
9:15 - 9:30 & Male-driven evolution of self-compatibility in diploid and polyploid Arabidopsis \\ 
 &  \\ 
9:30 - 9:45 & Allopolyploidy, diversification, and the Miocene grassland expansion \\ 
 &  \\ 
 & \textbf{Session 6} \\ 

8.30 - 8:45 & Genome Wide Association Mapping to examine genetic basis of quantitative traits in natural populations \\ 
 &  \\ 
8:45 - 9:00 & Local adaptation in herbivore feeding preferences: a marine-terrestrial contrast \\ 
 &  \\ 
9:00 - 9:15 & Asymmetric mismatch in secondary genital morphology increases harm to Drosophila females \\ 
 &  \\ 
9:15 - 9:30 & The Contribution of Rare and Common Variants to Standing Variation for Quantitative Traits in Capsella grandiflora \\ 
 &  \\ 
9:30 - 9:45 & 'Engine for speciation'? Experimental alteration of sexual conflict shows evidence of reproductive isolation in Drosophila Melanogaster. \\ 
 &  \\ 
 & \textbf{Session 7} \\ 

8.30 - 8:45 & Battle of the sexes: may the best fly win in reproduction \\ 
 &  \\ 
8:45 - 9:00 & Comparative analyses of biomechanical reproductive traits in harvestmen support intersexual coevolution via simultaneous mechanisms \\ 
 &  \\ 
9:00 - 9:15 & Ecological divergence, adaptive radiation and the evolution of sexual signaling traits in a complex of Australian agamid lizards \\ 
 &  \\ 
9:15 - 9:30 & Does sexual selection reinforce or impede responses to competitively-mediated disruptive selection? \\ 
 &  \\ 
 & \textbf{Session 8} \\ 

8.30 - 8:45 & Big groups, bad eggs and biogeography: regional and global patterns of brood parasitism’s effect on cooperative breeding \\ 
 &  \\ 
8:45 - 9:00 & How is geographic variation within species related to macroevolutionary patterns between species? \\ 
 &  \\ 
9:00 - 9:15 & The vanishing refuge revisited \\ 
 &  \\ 
9:15 - 9:30 & A Nearly Neutral Theory of Ecology and Macroevolution \\ 
 &  \\ 
9:30 - 9:45 & Connecting macroevolution to the genetics of adaptation: a case study using stomatal ratio \\ 
 &  \\ 
 & \textbf{Session 9} \\ 

8.30 - 8:45 & The Diversification of Insects, a phylogenetic perspective \\ 
 &  \\ 
8:45 - 9:00 & Divergence in life-cycle associated with variation in circadian genes in the European corn borer \\ 
 &  \\ 
9:00 - 9:15 & Genome-wide rates of molecular evolution are higher in mutualistic plant-nesting ants \\ 
 &  \\ 
9:15 - 9:30 & Life History Response to Juvenile Stress in Apis Mellifera \\ 
 &  \\ 
9:30 - 9:45 & Differential expression of carotenoid biosynthesis genes may underlie function in gall midges \\ 
 &  \\ 
 & \textbf{Session 10} \\ 

8.30 - 8:45 & Redefining the context in context-dependent mutation \\ 
 &  \\ 
8:45 - 9:00 & The effect of genetic quality on the mutation rate, estimated from genome sequences of mutation accumulation lines \\ 
 &  \\ 
9:00 - 9:15 & Patterns of amino acid sequence evolution across various time scales in the mitochondrial genomes of sexual and asexual snails \\ 
 &  \\ 
9:15 - 9:30 & The role of induced mutations as steps toward a Fisherian optimum in Arabidopsis thaliana under field conditions \\ 
 &  \\ 
9:30 - 9:45 & Susceptibility of Caenorhabditis elegans to a bacterial pathogen is a typical quantitative trait with an atypical mutational bias \\ 
 &  \\ 
 & \textbf{Session 11} \\ 

8.30 - 8:45 & Paternity in the NGS era \\ 
 &  \\ 
8:45 - 9:00 & Genome evolution and transposable element dynamics in wild sunflower species \\ 
 &  \\ 
9:00 - 9:15 & Transposable element and genome size evolution in sexual and functionally asexual evening primroses \\ 
 &  \\ 
9:15 - 9:30 & Next generation sequencing provides genetic insight to sapsucker speciation \\ 
 &  \\ 
9:30 - 9:45 & Adapterama @ BadDNA.org - DNA sequencing sample prep for Illumina instruments made easy (amplicons, RADseq, sequence capture & genomes) \\ 
 &  \\ 
 & \textbf{Session 12} \\ 

8.30 - 8:45 & An integrated phylogenomic approach toward pinpointing the origin of mitochondria \\ 
 &  \\ 
8:45 - 9:00 & The Genome of the Ctenophore Mnemiopsis leidyi: Bringing Resolution to the Phylogenetic Position of the Ctenophores \\ 
 &  \\ 
9:00 - 9:15 & Phylogeny and biogeography of two North American lamioid mint lineages (Lamiaceae) \\ 
 &  \\ 
9:15 - 9:30 & Early Evolution of the Genetic Basis for Soma in the Volvocine Green Algae \\ 
 &  \\ 
 & \textbf{Session 13} \\ 

8.30 - 8:45 & Simulation tests of probabilistic models for historical biogeography: DEC and DEC+J \\ 
 &  \\ 
8:45 - 9:00 & The Impact of the Rate Prior on Bayesian Estimation of Divergence Times with Multiple Loci \\ 
 &  \\ 
9:00 - 9:15 & Improving model-based phylogeographic inference by developing new spatially-explicit summary statistics \\ 
 &  \\ 
9:15 - 9:30 & Inferring Very Recent Population Growth Rate from Population-Scale Sequencing Data Using a Large-Sample Coalescent Estimator \\ 
 &  \\ 
9:30 - 9:45 & Extending BAMM: a computer program for analyzing complex macroevolutionary dynamics on phylogenetic trees \\ 
 &  \\ 
 & \textbf{Session 14} \\ 

8.30 - 8:45 & Assessing the Intraspecific Systematics of the Cotton Mouse, Peromyscus gossypinus, Using A Highly Variable Region of the Mitochondrial Genome \\ 
 &  \\ 
8:45 - 9:00 & Phylogenomics of deep-sea octocorals: new approaches to solve long-standing problems \\ 
 &  \\ 
9:00 - 9:15 & Convergence of nodes and internodes: Assessing the monophyly of bamboo corals (Cnidaria, Octocorallia, Isididae) and their diversity in the deep sea \\ 
 &  \\ 
9:15 - 9:30 & The Ruby Seadragon, a spectacular new species of seadragon (Syngnathidae) \\ 
 &  \\ 
9:30 - 9:45 & Genome-wide RAD data yields fine resolution species relationships in embiotocid surfperches \\ 
 &  \\ 
\end{longtabu}
\section{Saturday, May 23, 10:30 - 11:45}
\begin{longtabu} to \textwidth {lX}
 & \textbf{Session 1} \\ 

10:00 - 10:15 & Analysis of a multi gene phylogeny  Discriminating between alternative hypotheses and phylogenetic noise \\ 
 &  \\ 
10:15 - 10.30 & Towards inferring the history of life in the presence of lateral gene transfers \\ 
 &  \\ 
10.30 - 10.45 & Inferring the phylogenetic relationships of early dipteran lineages based on more than 1 000 orthologous genes from transcriptome data \\ 
 &  \\ 
10:45 - 11:00 & Nextgen data and phylogenetics   can we ignore incomplete lineage sorting and gene tree species tree conflict in phylogenetics \\ 
 &  \\ 
11:00 - 11:15 & Approaches to reducing spurious signal in phylogenomic datasets \\ 
 &  \\ 
 & \textbf{Session 2} \\ 

10:00 - 10:15 & An altitudinal cline in an ultraviolet floral trait is associated with changes in selection and pollination context \\ 
 &  \\ 
10:15 - 10.30 & Identification of major QTLs underlying floral pollination syndrome divergence in Penstemon \\ 
 &  \\ 
10.30 - 10.45 & Making sense of floral scents  floral scent in the genus Mimulus and its role in pollinator shifts \\ 
 &  \\ 
10:45 - 11:00 & Explorations of parallel evolution of hummingbird pollination in Mimulus \\ 
 &  \\ 
11:00 - 11:15 & Evolution of hawkmoth pollination in the gourd family  Cucurbitaceae \\ 
 &  \\ 
 & \textbf{Session 3} \\ 

10:00 - 10:15 & Testing models of recent demographic history with genome scale data  a case study of the meltwater stonefly  Lednia tumana \\ 
 &  \\ 
10:15 - 10.30 & Determining asexual versus sexual propagation in the octocoral Paramuricea using RAD sequencing \\ 
 &  \\ 
10.30 - 10.45 & The co evolution of altruism and collective movement \\ 
 &  \\ 
10:45 - 11:00 & Reconstructing the Demographic History of Ad lie Penguins  Pygoscelis adeliae  Using mtDNA and Coalescent Methods \\ 
 &  \\ 
11:00 - 11:15 & Assessment of the subspecies status of Calidris maritima littoralis    Aves  Charidriiformes  Scolopacidae \\ 
 &  \\ 
 & \textbf{Session 4} \\ 

10:00 - 10:15 & Evolving isolation mechanisms between two incipient species of an endoparasitic wasp \\ 
 &  \\ 
10:15 - 10.30 & The predictors of avian song evolution  sexual selection and the trade off between acoustic and visual signals \\ 
 &  \\ 
10.30 - 10.45 & Morphological and genetic analyses of interspecific introgression in a natural damselfly population \\ 
 &  \\ 
10:45 - 11:00 & Evolution of an integrated sexual display  genomic analysis of behavioral and morphological components of courtship song in a Hawaiian cricket \\ 
 &  \\ 
11:00 - 11:15 & Evolution of female song production in Drosophila virilis group species \\ 
 &  \\ 
 & \textbf{Session 5} \\ 

10:00 - 10:15 & Comparative seascape genetics of coral reef fishes  integrating genetic datasets and biophysical models across a common seascape \\ 
 &  \\ 
10:15 - 10.30 & Genetic basis of adaptive behavior  do proximate genetic mechanisms suggest evolutionary causes \\ 
 &  \\ 
10.30 - 10.45 & The genetic basis of environmental adaptation in house mice \\ 
 &  \\ 
10:45 - 11:00 & Intraguild predation results in genome wide adaptation in the Threespine Stickleback \\ 
 &  \\ 
11:00 - 11:15 & The Genomic Architecture of Adaptive Quantitative Trait Variation in Darwin s Finches \\ 
 &  \\ 
 & \textbf{Session 6} \\ 

10:00 - 10:15 & The facultative symbiont Rickettsia protects whiteflies against cryptic Pseudomonas syringae pathogens \\ 
 &  \\ 
10:15 - 10.30 & Evidence of pathogen induced recombination among low fitness lineages of Drosophila melanogaster \\ 
 &  \\ 
10.30 - 10.45 & Genomic response to 30 years of selection for increased lifespan reveals increased immunity as correlated trait \\ 
 &  \\ 
10:45 - 11:00 & Intrahost HIV Evolution During Early Infection \\ 
 &  \\ 
11:00 - 11:15 & Evolution of immune response in Drosophila melanogaster populations selected against a gram negative bacteria \\ 
 &  \\ 
 & \textbf{Session 7} \\ 

10:00 - 10:15 & Environmental stability  niche conservatism  and energetic constraints  explaining tropical biodiversity \\ 
 &  \\ 
10:15 - 10.30 & Testing the role of ecology and life history in structuring genetic variation across a landscape  A comparative ecophylogeographic approach \\ 
 &  \\ 
10.30 - 10.45 & Incorporating Evolutionary History into Ecological Niche Modeling \\ 
 &  \\ 
10:45 - 11:00 & Spatial niches influence biodiversity during adaptive radiation \\ 
 &  \\ 
11:00 - 11:15 & The coupling of niche divergence and lineage diversification at different spatial scales \\ 
 &  \\ 
 & \textbf{Session 8} \\ 

10:00 - 10:15 & Drought tolerance of locally adapted Arabidopsis thaliana \\ 
 &  \\ 
10:15 - 10.30 & Do effects of nutritional stress on reproductive traits translate from lab to field \\ 
 &  \\ 
10.30 - 10.45 & Investigation of salt tolerance in an association mapping population of cultivated sunflower  Helianthus annuus L \\ 
 &  \\ 
10:45 - 11:00 & The genetics of adaptation to a granite outcrop environment in the Mimulus guttatus species complex \\ 
 &  \\ 
11:00 - 11:15 & Life history evolution of serotinous trees  the role of inter fire recruitment and dispersal \\ 
 &  \\ 
 & \textbf{Session 9} \\ 

10:00 - 10:15 & One species or three  Resolving lineage boundaries in dusky salamanders through sexual isolation trials and next generation sequencing \\ 
 &  \\ 
10:15 - 10.30 & Using sequence capture to reconstruct North American mammoth phylogeny and phylogeography \\ 
 &  \\ 
10.30 - 10.45 & Variation in Interploid Reproductive Isolation within the Campanula rotundifolia Polyploid Complex \\ 
 &  \\ 
10:45 - 11:00 & Disentangling phylogenetic relationships complicated by polyploidy in the genus Phlox  Polemoniaceae \\ 
 &  \\ 
11:00 - 11:15 & Inferential Evolution and the Reflection Principle \\ 
 &  \\ 
 & \textbf{Session 10} \\ 

10:00 - 10:15 & Dynamics of evolutionary innovations in cancer \\ 
 &  \\ 
10:15 - 10.30 & The fates of rare beneficial lineages in hypermutable asexual populations \\ 
 &  \\ 
10.30 - 10.45 & Accurate detection of mutations from short read sequencing data \\ 
 &  \\ 
10:45 - 11:00 & Genetic constraints cause mutation rate catastrophe \\ 
 &  \\ 
11:00 - 11:15 & Experimental evidence for both upward instability and decline of the genomic mutation rate across isogenic asexual Escherichia coli populations \\ 
 &  \\ 
 & \textbf{Session 11} \\ 

10:00 - 10:15 & Independent origins of the avian Z chromosome reveal contrasting short  and long term dynamics of sex specific selection \\ 
 &  \\ 
10:15 - 10.30 & Do interpopulation crosses and genetic drift disrupt sex determination in Tigriopus californicus  a species with polygenic sex determination \\ 
 &  \\ 
10.30 - 10.45 & Sexual co adaptation or conflict  Are genes that interact with the mitochondria more often on the sex chromosomes \\ 
 &  \\ 
10:45 - 11:00 & Sex Chromosome Dosage Compensation in Heliconius butterflies \\ 
 &  \\ 
11:00 - 11:15 & Molecular basis of protogynous sex change in fish \\ 
 &  \\ 
 & \textbf{Session 12} \\ 

10:00 - 10:15 & Exploring patterns of symbiont diversity in natural pea aphid populations \\ 
 &  \\ 
10:15 - 10.30 & Environmental context matters  the impact of microbial symbiont on invasive insect host Megacopta cribraria is mediated by host plant \\ 
 &  \\ 
10.30 - 10.45 & Holarctic biogeography of a widespread host symbiont association \\ 
 &  \\ 
10:45 - 11:00 & Interactions between host phylogeny and biogeography structure sponge associated microbial communities \\ 
 &  \\ 
11:00 - 11:15 & Host evolution and ecology govern community assembly of the gut microbiome in lemurs \\ 
 &  \\ 
 & \textbf{Session 13} \\ 

10:00 - 10:15 & MIGRATE 40  many loci  divergences  and assignments \\ 
 &  \\ 
10:15 - 10.30 & General extensions of Qst Fst for detecting adaptation in quantitative traits \\ 
 &  \\ 
10.30 - 10.45 & Evaluating methods for estimating effective population size in the presence of migration \\ 
 &  \\ 
10:45 - 11:00 & Parallel MCMC and Inference of Ancient Demography under the Isolation with Migration  IM  Model \\ 
 &  \\ 
11:00 - 11:15 & Life history  Selection and Effective Population Size shaping Evolution during Colonization   Lessons from Drosophila melanogaster \\ 
 &  \\ 
 & \textbf{Session 14} \\ 

10:00 - 10:15 & Developmental plasticity and reproductive fitness in the house mouse \\ 
 &  \\ 
10:15 - 10.30 & Tinkering with the axial skeleton  vertebral number variation in ecologically divergent threespine stickleback populations \\ 
 &  \\ 
10.30 - 10.45 & The evolution of Acacia and biogeographic connections of the Australian continent \\ 
 &  \\ 
10:45 - 11:00 & The role of extinction in the assembly of large scale biodiversity patterns \\ 
 &  \\ 
11:00 - 11:15 & Cavity nesting makes flycatchers fecund and fly farther  evolutionary links between cavity nesting  clutch size and migration in the Muscicapidae \\ 
 &  \\ 
\end{longtabu}


\end{document}
