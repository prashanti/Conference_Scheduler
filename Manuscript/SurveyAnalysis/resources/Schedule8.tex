

\documentclass{article}
\usepackage[utf8]{inputenc}
\usepackage{array}
\usepackage{ltxtable, tabularx, longtable,tabu}
\title{Schedule}
\date{}

\begin{document}
\maketitle{}

\section{Saturday, May 23, 8:30 - 9:45}
\begin{longtabu} to \textwidth {lX}
 & \textbf{Session 1} \\ 

8.30 - 8:45 & The consequences of pollinator declines on the quantity and quality of offspring in two New Zealand tree species \\ 
 &  \\ 
8:45 - 9:00 & Joint effects of pollen limitation and pollen competition on offspring quality in a wind-pollinated herb \\ 
 &  \\ 
9:00 - 9:15 & Pollen competition in style: Is bigger always better? \\ 
 &  \\ 
9:15 - 9:30 & Adaptive floral traits: selection through male, female, and total fitness \\ 
 &  \\ 
 & \textbf{Session 2} \\ 

8.30 - 8:45 & The plastomes of mycoheterotrophic Ericaceae exhibit extensive gene loss and rearrangements \\ 
 &  \\ 
8:45 - 9:00 & Dad saves the day: biparental plastid inheritance rescues cytonuclear incompatibility \\ 
 &  \\ 
9:00 - 9:15 & Contrasting patterns of plastid and mitochondrial genetic diversity in gynodioecious Lobelia siphilitica (Campanulaceae) \\ 
 &  \\ 
9:15 - 9:30 & Organellar phylogenomics of green plants \\ 
 &  \\ 
9:30 - 9:45 & Horizontal gene transfer in the mitochondrial genome of Monsonia emarginata \\ 
 &  \\ 
 & \textbf{Session 3} \\ 

8.30 - 8:45 & Computing the Quartet Distance for Sets of Heterogeneous Phylogenetic Trees \\ 
 &  \\ 
8:45 - 9:00 & guenomu: a Bayesian supertree program for species tree reconstruction \\ 
 &  \\ 
9:00 - 9:15 & Mean and Variance of Phylogenetic Trees \\ 
 &  \\ 
9:15 - 9:30 & Polymorphism-Aware Phylogenetic Model (PoMo): An allele frequency-based approach for species tree estimation \\ 
 &  \\ 
 & \textbf{Session 4} \\ 

8.30 - 8:45 & Disentangling phylogenetic relationships in an explosive bird radiation \\ 
 &  \\ 
8:45 - 9:00 & Phylogeny, morphology and ontogeny of the Spikethumb Frogs (Hylidae: Plectrohyla) \\ 
 &  \\ 
9:00 - 9:15 & Increasing phylogenetic resolution in a hyperdiverse radiation of blood feeding flies \\ 
 &  \\ 
9:15 - 9:30 & Phylogenomic analysis of yellowjackets and hornets (Hymenoptera, Vespidae) \\ 
 &  \\ 
 & \textbf{Session 5} \\ 

8.30 - 8:45 & Insight into the speciation process: patterns of reproductive isolation in five stickleback species pairs that span the speciation continuum \\ 
 &  \\ 
8:45 - 9:00 & Genome divergence at the onset of speciation with gene flow \\ 
 &  \\ 
9:00 - 9:15 & What drives genetic and phenotypic divergence for Iris hexagona \\ 
 &  \\ 
9:15 - 9:30 & Major Ecological Shifts both Promote and Retard Speciation in Timema Stick Insects \\ 
 &  \\ 
 & \textbf{Session 6} \\ 

9:30 - 9:45 & Coupling Between Protein Level Selection and Codon Usage Optimization in the Evolution of Bacteria and Archaea \\ 
 &  \\ 
8.30 - 8:45 & Characterizing Independent Adaptive Mutations in Yeast Experimental Evolution Using DNA Barcodes \\ 
 &  \\ 
8:45 - 9:00 & Into the ant nest: molecular evolution of chemoreception and host specialization in predatory paussine beetles \\ 
 &  \\ 
9:00 - 9:15 & Uncovering genome-wide targets of convergent evolution along a re-established flowering time cline in the introduced range of Arabidopsis thaliana \\ 
 &  \\ 
9:15 - 9:30 & Faster rates in snakes? Molecular evolution of the insulin signaling pathway in amniotes \\ 
 &  \\ 
 & \textbf{Session 7} \\ 

8.30 - 8:45 & Explaining the novel axes of adaptive phenotypic diversification in complex societies using the turtle ants \\ 
 &  \\ 
8:45 - 9:00 & Stepwise evolution of social complexity in ground-dwelling squirrels \\ 
 &  \\ 
9:00 - 9:15 & Looking for signatures of social conflict in secondary metabolites of cooperative amoebae \\ 
 &  \\ 
9:15 - 9:30 & Periodic Social Niche construction \\ 
 &  \\ 
9:30 - 9:45 & Fairness and Wisdom: Paths to Leadership \\ 
 &  \\ 
 & \textbf{Session 8} \\ 

8.30 - 8:45 & The role of hybrid incompatibilities in hybrid zone structure \\ 
 &  \\ 
8:45 - 9:00 & A Model of Genome-Wide Patterns of Ancestry in a Secondary Contact Zone \\ 
 &  \\ 
9:00 - 9:15 & Evolutionary origins and genomic consequences of hybridogenesis in Pogonomyrmex harvester ants \\ 
 &  \\ 
9:15 - 9:30 & Hybrid Zones, Genomic Ancestry Patterns, and Genomic Scans for Hybrid Sterility Genes \\ 
 &  \\ 
9:30 - 9:45 & Shaking the parrotfish tree: hybridization in a peripheral environment produces phenotypic novelty \\ 
 &  \\ 
 & \textbf{Session 9} \\ 

8.30 - 8:45 & Obligate insect endosymbionts exhibit increased ortholog length variation and loss of large accessory proteins concurrent with genome shrinkage. \\ 
 &  \\ 
8:45 - 9:00 & Cytotype growth response to phosphorus limitation and arbuscular mycorhizal colonization in  Chamerion angustifolium \\ 
 &  \\ 
9:00 - 9:15 & Developmental integration of an obligate intracellular symbiont \\ 
 &  \\ 
9:15 - 9:30 & Allele changes during spore formation on the mycorrhizal fungi, Rhizophagus irregularis \\ 
 &  \\ 
9:30 - 9:45 & Fitness feedbacks and alignment of interests in mutualisms \\ 
 &  \\ 
 & \textbf{Session 10} \\ 

8.30 - 8:45 & Toxin gene-expression variation in the eastern diamondback rattlesnake (Crotalus adamanteus):  adaptive or neutral? \\ 
 &  \\ 
8:45 - 9:00 & All venoms are not created equal: the distribution and adaptive significance of venom types in the Timber Rattlesnake (Crotalus horridus) \\ 
 &  \\ 
9:00 - 9:15 & Evolution of the venom gland transcriptome in widow spiders \\ 
 &  \\ 
9:15 - 9:30 & Uncovering venom neurotoxin gene family evolution from black widow and house spider genomes and transcriptomes \\ 
 &  \\ 
9:30 - 9:45 & The secrets of staying young: Investigating the evolution of venom neoteny in the Timber Rattlesnake, Crotalus horridus. \\ 
 &  \\ 
 & \textbf{Session 11} \\ 

8.30 - 8:45 & Phylogenetic Skew: A New Index of Community Diversity \\ 
 &  \\ 
8:45 - 9:00 & A new dynamic model for the phylogenetic assembly of the ecological community \\ 
 &  \\ 
9:00 - 9:15 & High-throughput sequencing characterization of meiofaunal communities in northern Gulf of Mexico \\ 
 &  \\ 
 & \textbf{Session 12} \\ 

8.30 - 8:45 & The role of deleterious mutations in influenza s antigenic evolution \\ 
 &  \\ 
8:45 - 9:00 & Intrahost competition in mixed-strain malaria infections may slow the evolution of resistance in high-transmission settings \\ 
 &  \\ 
9:00 - 9:15 & The Contribution of Migration and Mutation to the Population Shift Following Widespread Rotavirus Vaccination in the United States \\ 
 &  \\ 
9:15 - 9:30 & Signatures of Selection on RNA Structures in Influenza Genomes \\ 
 &  \\ 
9:30 - 9:45 & Temporal Dynamics of a Ranavirus Outbreak in Chelonians with Mosquitoes as Possible Vectors \\ 
 &  \\ 
 & \textbf{Session 13} \\ 

8.30 - 8:45 & Why wait? The optimal waiting time between an environmental cue and a plastic response \\ 
 &  \\ 
8:45 - 9:00 & Soil microbial communities cause differential selection and plasticity of flowering time in the wild mustard Boechera stricta \\ 
 &  \\ 
9:00 - 9:15 & Predator-induced phenotypic plasticity within- and across-generations \\ 
 &  \\ 
9:15 - 9:30 & Evidence for natural selection of phentoypic plasticity in Plantago lanceolata \\ 
 &  \\ 
9:30 - 9:45 & Plasticity in offspring growth maximizes potential fitness in unpredictable environments \\ 
 &  \\ 
 & \textbf{Session 14} \\ 

8.30 - 8:45 & Warning signals are seductive: Relative contributions of color and pattern cues to predator avoidance and mate attraction in Heliconius butterflies \\ 
 &  \\ 
8:45 - 9:00 & Ecological variation affects signal reliability in Gambusia hubbsi \\ 
 &  \\ 
9:00 - 9:15 & A tradeoff between natural and sexual selection underlies evolution of sexual signal diversity in Bahamas mosquitofish \\ 
 &  \\ 
9:15 - 9:30 & Directional Selection on Aposematic Coloration in the Dyeing Poison Frog (Dendrobates tinctorius) \\ 
 &  \\ 
9:30 - 9:45 & Parasite-mediated sexual signaling: what do females gain? \\ 
 &  \\ 
\end{longtabu}
\section{Saturday, May 23, 10:30 - 11:45}
\begin{longtabu} to \textwidth {lX}
 & \textbf{Session 1} \\ 

10:00 - 10:15 & Hungry mothers ability to abort and cannibalize their offspring enables the evolution of placentotrophy in a lizard \\ 
 &  \\ 
10:15 - 10.30 & Selfing allele favored by male function in experimental populations of Witheringia solanacea in Costa Rica \\ 
 &  \\ 
10.30 - 10.45 & Impact of self fertilization on fecundity  the timing of reproduction  and population genetic structure  in a marine ribbon worm  Nemertea \\ 
 &  \\ 
10:45 - 11:00 & Partitioning additive and nonadditive genetic effects on offspring quality in a broadcast spawning marine invertebrate \\ 
 &  \\ 
11:00 - 11:15 & Plasticity in offspring growth maximizes potential fitness in unpredictable environments \\ 
 &  \\ 
 & \textbf{Session 2} \\ 

10:00 - 10:15 & Analysis of variance in fitness over 50 000 generations in an evolution experiment \\ 
 &  \\ 
10:15 - 10.30 & Condition dependent alternative reproductive tactics in territorial damselflies  the role of wing shape in territory holding potential \\ 
 &  \\ 
10.30 - 10.45 & Historical responses of Antarctic penguins and seals to climate change \\ 
 &  \\ 
10:45 - 11:00 & Fitness feedbacks and alignment of interests in mutualisms \\ 
 &  \\ 
11:00 - 11:15 & Fitness functions and distributions  the shape of things to come \\ 
 &  \\ 
 & \textbf{Session 3} \\ 

10:00 - 10:15 & Evolution of sexual dimorphism within puppet beetles \\ 
 &  \\ 
10:15 - 10.30 & Sexual selection and sperm competition in a widespread dung fly  Sepsis punctum  Diptera  Sepsidae \\ 
 &  \\ 
10.30 - 10.45 & Sexually selected sperm competition genes also contribute to postmating species barriers  conspecific sperm precedence  between D melanogaster and D \\ 
 &  \\ 
10:45 - 11:00 & Testing Darwin s hypothesis on the evolution of ornamental eyespots in peafowl and their relatives \\ 
 &  \\ 
11:00 - 11:15 & The role of selection in the rapid evolution of reproductive genes  are we using the correct null hypothesis \\ 
 &  \\ 
 & \textbf{Session 4} \\ 

10:00 - 10:15 & Bayesian Co estimation of Selfing Rate and Locus Specific Mutation Rates for a Partially Selfing Population \\ 
 &  \\ 
10:15 - 10.30 & The evolution of selfing is accompanied by an increased frequency of effectively neutral and strongly deleterious mutations \\ 
 &  \\ 
10.30 - 10.45 & Germline mutation as an outcome of evolved and continually evolving biological processes \\ 
 &  \\ 
10:45 - 11:00 & Characterizing independent adaptive mutations in yeast experimental evolution using DNA barcodes \\ 
 &  \\ 
11:00 - 11:15 & The role of deleterious mutations in influenza s antigenic evolution \\ 
 &  \\ 
 & \textbf{Session 5} \\ 

10:00 - 10:15 & Comparative genomics and transcriptomics of the New Zealand Giant Weta \\ 
 &  \\ 
10:15 - 10.30 & Understanding genome evolution in the dogwood genus Cornus L  Cornaceae  from analyses of transcriptome sequences \\ 
 &  \\ 
10.30 - 10.45 & Using transcriptomes for functional phylogenomic studies  promises and pitfalls \\ 
 &  \\ 
10:45 - 11:00 & aTRAM   automated Target Restricted Assembly Method   A fast method for assembling genes from massively parallel sequence data \\ 
 &  \\ 
11:00 - 11:15 & Assembling genes without genomes  phylogenomic exploration within the family Salicaceae \\ 
 &  \\ 
 & \textbf{Session 6} \\ 

10:00 - 10:15 & Physiological divergence as a driver of the Anolis adaptive radiation \\ 
 &  \\ 
10:15 - 10.30 & Explaining the novel axes of adaptive phenotypic diversification in complex societies using the turtle ants \\ 
 &  \\ 
10.30 - 10.45 & Ecological controls of mammalian diversification \\ 
 &  \\ 
10:45 - 11:00 & Does Ecological Speciation Explain the Origin of Tropical Savanna Woody Flora \\ 
 &  \\ 
11:00 - 11:15 & Sex determination and the birth and death of species \\ 
 &  \\ 
 & \textbf{Session 7} \\ 

10:00 - 10:15 & Using the multi species allele frequency spectrum  msAFS  for next generation comparative phylogeography \\ 
 &  \\ 
10:15 - 10.30 & Performance and utility of Single Nucleotide Polymorphisms  SNPs  in fine scale population study \\ 
 &  \\ 
10.30 - 10.45 & Inferring allele frequency trajectories of experimentally evolved Drosophila populations with Gaussian process models \\ 
 &  \\ 
10:45 - 11:00 & Does Energy Availability Predict Gastropod Reproductive Strategies \\ 
 &  \\ 
11:00 - 11:15 & Comparative Phylogeography of Lizards from the Brazilian Cerrado \\ 
 &  \\ 
 & \textbf{Session 8} \\ 

10:00 - 10:15 & The Effect of Parasites on the Ability of Self Fertilization to Invade Outcrossing Host Populations \\ 
 &  \\ 
10:15 - 10.30 & Empirical evidence for a continuum between semelparity and iteroparity \\ 
 &  \\ 
10.30 - 10.45 & Parasite mediated sexual signaling  what do females gain \\ 
 &  \\ 
10:45 - 11:00 & Life cycle evolution and wnt signaling in the Hydractiniidae  Cnidaria  Hydrozoa \\ 
 &  \\ 
11:00 - 11:15 & The evolution of semelparity and egg size \\ 
 &  \\ 
 & \textbf{Session 9} \\ 

10:00 - 10:15 & Estimating how contemporary taxa will evolve in the future to understand how island communities were assembled in the past \\ 
 &  \\ 
10:15 - 10.30 & Ecological Correlates of Body Size Change in Island Populations of Wild House Mice \\ 
 &  \\ 
10.30 - 10.45 & Estimating gene flow directionality and demography of western Atlantic Syngnathidae with population genomic RADseq data \\ 
 &  \\ 
10:45 - 11:00 & The phylogeography of Peromyscus maniculatus across the northern California Channel Islands \\ 
 &  \\ 
11:00 - 11:15 & Resolving the complex evolutionary history of a Philippine songbird with genome wide markers \\ 
 &  \\ 
 & \textbf{Session 10} \\ 

10:00 - 10:15 & Analysis of nuclear and mitochondrial DNA reveals cryptic speciation in North American flying squirrels  Glaucomys \\ 
 &  \\ 
10:15 - 10.30 & Assessing GIS estuarine habitat predictions with a new statistical approach for genetic signatures of postglacial recolonization \\ 
 &  \\ 
10.30 - 10.45 & Measuring phylogenetic conservatism of extinction in vertebrates  deep time signals and methods \\ 
 &  \\ 
10:45 - 11:00 & Cenozoic mammals and the biology of extinction \\ 
 &  \\ 
11:00 - 11:15 & Fairness and wisdom  the emergence of leadership \\ 
 &  \\ 
 & \textbf{Session 11} \\ 

10:00 - 10:15 & ddRAD seq analyses of population structure in brood parasitic indigobirds  Vidua spp \\ 
 &  \\ 
10:15 - 10.30 & Pyrosequencing of surface waters in the English Channel reveals novel early diverging fungal diversity \\ 
 &  \\ 
10.30 - 10.45 & Comparative analysis of preference and performance genes in the evolution of host specialization in Neodiprion sawflies \\ 
 &  \\ 
10:45 - 11:00 & Pathogen Host Interactions within Freshwater Systems \\ 
 &  \\ 
11:00 - 11:15 & Evolution of wild yeasts as opportunistic pathogens during experimental co infection \\ 
 &  \\ 
 & \textbf{Session 12} \\ 

10:00 - 10:15 & Strong selection drives reinforcement in Phlox \\ 
 &  \\ 
10:15 - 10.30 & Do placentation and superfetation facilitate living life in the fast lane  A preliminary field study \\ 
 &  \\ 
10.30 - 10.45 & EDA signaling and phenotypic evolution of sculpins  Cottus  after admixture \\ 
 &  \\ 
10:45 - 11:00 & Clinal variation in floral color and flavonoids along alpine elevation gradients in Silene vulgaris \\ 
 &  \\ 
11:00 - 11:15 & Understanding epistasis and gene networks in complex traits  An analysis of aggression in a model system \\ 
 &  \\ 
 & \textbf{Session 13} \\ 

10:00 - 10:15 & Uncovering genome wide targets of convergent evolution along a re established flowering time cline in the introduced range of Arabidopsis thaliana \\ 
 &  \\ 
10:15 - 10.30 & Soil microbial communities cause differential selection and plasticity of flowering time in the wild mustard Boechera stricta \\ 
 &  \\ 
10.30 - 10.45 & Building developmental mechanisms into genotype phenotype predictions in changing environments \\ 
 &  \\ 
10:45 - 11:00 & Evolutionary responses of plants to urban environments \\ 
 &  \\ 
11:00 - 11:15 & A within species comparative method reveals reproductive character displacement to a geographic mosaic of interspecific interactions \\ 
 &  \\ 
 & \textbf{Session 14} \\ 

10:00 - 10:15 & Coevolution of Complexity as Seen by a Digital Hosts Adaptive Landscape \\ 
 &  \\ 
10:15 - 10.30 & Molecular systematics of the North American tiger salamander radiation using parallel tagged amplicon sequence data \\ 
 &  \\ 
10.30 - 10.45 & The evolution of placentae  complex trait evolution can be constrained by ancient features of an organism s genome \\ 
 &  \\ 
10:45 - 11:00 & Extending the concept of diversity partitioning to characterize phenotypic complexity \\ 
 &  \\ 
11:00 - 11:15 & Cytotype growth response to phosphorus limitation and arbuscular mycorhizal colonization in  Chamerion angustifolium \\ 
 &  \\ 
\end{longtabu}

\end{document}
